\documentclass{article}

\usepackage{cheatsheet}
\usepackage[german]{babel}

\begin{document}
\noindent
\begin{card}
  \cardtitle{Hypoglykämie}{\small\erw\footnotesize\kind}
  \cardmeta{RLP\\2.01}

  \note{Notes}{%
    \begin{itemize}[twocol]
      \item \textbf{Sicherer i.v. Zugang!}
      \item Glukose langsam applizieren
      \item \textbf{persönliche} Übergabe an \\ Hausarzt, ÄBP, Klinik
    \end{itemize}
  }

  \note{Definition}{%
    \vspace{-1em}
    \begin{center}
      {\kind}: Gewicht < \SI{40}{\kilo\gram}
      \hspace{2em}
      {\erw}: Gewicht $\geq$ \SI{40}{\kilo\gram}
    \end{center}
  }

  \drug{Glukose}{}{%
    \dosierung{%
      {\erw}: \SI{8}{\gram} i.v. + \SI{8}{\gram} Infusion (+ \wdh{} \SI{8}{\gram} i.v.)
      \\[0.5em]
      {\kind}: \SI{0.2}{\gram\per\kgKG} i.v. + \SI{0.2}{\gram\per\kgKG} Infusion (+ \wdh{} \SI{0.2}{\gram\per\kgKG} i.v.)
    }
  }

  \note{{\kind} Verdünnung G-40}{%

    \begin{center}
      Glukose \SI{40}{\percent} mit NaCl 1:1 verdünnen \\[0.5em]
      $\Rightarrow$ \SI{0.2}{\gram\per\milli\liter} \\[1.5em]
      \textbf{\SI{1}{\milli\liter\per\kgKG} applizieren}
    \end{center}
  }

\end{card}
\end{document}
