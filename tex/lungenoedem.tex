\documentclass{article}

\usepackage{cheatsheet}
\usepackage[german]{babel}

\begin{document}
\noindent
\begin{card}
  \cardtitle{Kardiales Lungenödem}{\erw}
  \cardmeta{RLP\\2.1}

  \note{Notes}{%
    \begin{itemize}[twocol]
      \item \textbf{DD:} Asthma, COPD, Pneumonie
      \item feuchte Rasselgeräusche,\\evtl. schaumiger Auswurf, Unterschenkelödeme, Tachykardie
      \item Anamnese: Herzinsuffizienz, häufig Hypertonie und/oder KHK
      \item 12-Kanal EKG
      \item CPAP/ASB erwägen
    \end{itemize}
  }

  \drug{Nitrolingual}{Glyceroltrinitrat}{%
    \dosierung{%
      \SI{1}{\hub} s.l. (+ \wdh{1} | \SIrange{3}{5}{\minute})
    }
    \begin{kontraindikationen}
      \item RR$_\text{sys}$ < \SI{110}{\mmHg}
      \item akute Rechtsherzbelastung
      \item Einnahme PDE-5-Hemmer \\
        (Viagra {\zeit} < \SI{24}{\hour} | Cialis {\zeit} < \SI{36}{\hour})
    \end{kontraindikationen}
  }

  \drug{Lasix}{Furosemid}{%
    \dosierung{%
      \SI{40}{\milli\gram} i.v. (langsam!)
    }
    \begin{kontraindikationen}
      \item RR$_\text{sys}$ < \SI{110}{\mmHg}
      \item bekannte Hypokaliämie
      \item Blasenentleerungsstörung \\ (ohne Katheter)
    \end{kontraindikationen}
  }

% \vfill

\end{card}
\end{document}
