\documentclass{article}

\usepackage{cheatsheet}

\renewcommand{\sopName}{Analgesie}
\renewcommand{\sopVersion}{01-JAN-2017}
\renewcommand{\sopEditor}{(cwe)}


\begin{document}
\fftext%

\maketitle

\notes{%
  \begin{description}
    \item[Spastisch abdominal:] Butylscopolamin
    \item[Bewegungsapparat:] Paracetamol + Nalbuphin
  \end{description}
}

\vfill

\drug{Butylscopolamin}{%
  \dosierung{0.3 mg / kg KG (+ 1 Wdh.)}
  \begin{minipage}[b]{0.5\textwidth}
    \begin{kontraindikationen}
      \item VAS $\le$ 5
      \item Glaukom
      \item \textbf{Herzrythmusströrungen}
      \item Herzinsuffizienz
      \item Hyperthyreose
    \end{kontraindikationen}
  \end{minipage}
  \hfill
  \begin{minipage}[b]{0.5\textwidth}
    \begin{kontraindikationen}
      \item (\mann) Prostatavergrößerung
      \item (\frau) Gravida / Stillen
      \item hereditäre \\ Fructoseintoleranz
      \item Megacolon
    \end{kontraindikationen}
  \end{minipage}
}

\vfill

\drug{Paracetamol}{%
  \dosierung{15 mg / kg KG ({\kind} 11--50 kg) | 1000g ({\erw} $\ge$ 50 kg)}
  \begin{minipage}{0.5\textwidth}
    \begin{kontraindikationen}
      \item VAS $\le$ 5
      \item G6PDH-Mangel
      \item Nieren-/Leberstörung
      \item chron./akuter C$_2$-Abusus
      \item Antiepileptika
      \item Paracetamol {\uhr} < 4h
      \item Tageshöchstdosis \\ 70 mg/kg KG
    \end{kontraindikationen}
  \end{minipage}
  \hfill
  \begin{minipage}{0.5\textwidth}
    \centering
    \begin{tabular}{cr}
      \toprule
      \textbf{Gewicht} & \textbf{Dosis} \\
      \midrule
      15 kg & 225 mg \\
      20 kg & 300 mg \\
      25 kg & 375 mg \\
      30 kg & 450 mg \\
      35 kg & 525 mg \\
      40 kg & 600 mg \\
      45 kg & 675 mg \\
      50 kg & 1000 mg \\
      \bottomrule
    \end{tabular}
  \end{minipage}
}

\vfill

\drug{Nalbuphin}{%
  \dosierung{0.1 mg / kg KG (+ 1 Wdh.)}
  \begin{kontraindikationen}
    \item VAS $\le$ 5
    \item Nieren-/Leberstörung
    \item chron./akuter C$_2$-Abusus
    \item Opiatabhängigkeit
    \item Einnahme {\uhr} < 24h \\
      Opiaten, Antihistaminika, Antidepressiva, Sedativa
  \end{kontraindikationen}
}

\vfill

\end{document}
